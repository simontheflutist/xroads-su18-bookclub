\documentclass[outline]{bookclub}

\begin{document}
	\setcounter{chapter}{2}
	\chapter{July 22, 2018: II.~(vii)--III.~(vi)}
	\begin{multicols}{2}
		\begin{enumerate}
			\setcounter{enumi}{1}
			\item 
			\begin{enumerate}
				\setcounter{enumii}{6}
				\item Unhelpfulness of teaching an unfree will?
				\begin{mynotes}
					\item Background: even if the doctrine of an unfree will is \emph{correct}, it is, albeit lawful, perhaps not \emph{helpful} to proclaim it.
					\begin{response}
						\item Response: but we are accountable to all of Scripture, and even Paul and Jesus use such terms as Erasmus would reject.
					\end{response}
					\item Objection: it does not make sense that God should punish and reward \emph{his own} works in us.
					\begin{response}
						\item Response: if this is in fact the case, then we have no use to investigate why it is so (Rom.~9).
					\end{response}
					\item Objection: if God works in us good and evil, who would endeavor to reform his ways?
					\begin{response}
						\item Response: no one can, but this is the work of the Holy Spirit.
					\end{response}
					\item A realization of our far-reaching inability diminishes our pride and magnifies the active grace of God.
					\item The paradox of divine sovereignty and human responsibility is reconciled in God by \emph{faith}.
				\end{mynotes}
			\item Free will vs.~``\,`mere necessity'\,''
			\begin{mynotes}
				\item Necessity of human actions is not \emph{of compulsion} but of \emph{immutability}. (cf.~Aquinas on Hell-dwellers) The will naturally directed towards evil is directed by God towards good.
				\item The will is ``a beast standing between two riders,'' mastered by Satan until overcome by God.
			\end{mynotes}
			\item Erasmus's ``free will'' (``ineffective apart from grace'') is not free.
			\begin{mynotes}
				\item Erasmus confuses ``dispositional quality''/``passive aptitude,'' which has no need for grace, with the power to will good.
				\item A fully robust idea of human free will deifies man and dethrones God.
				\item We should not define ``free will'' weak enough to justify, then retain these conclusions when it is redefined to a theologically infeasible strength.
			\end{mynotes}
			\item \begin{mynotes}
				\item To preach ``Christ crucified'' consists in more than ``\,`Christ was crucified.'\,'' At free will the Gospel stands or falls.
			\end{mynotes}
			\end{enumerate}
			\item \begin{enumerate}
				\item Luther argues on the basis of authority alone?
				\begin{mynotes}
					\item The fathers' acts and words are to be judged by Scripture, and hold no authority in themselves.
					\item Free will is not attested by miracles of Scripture or history, and by itself seems to have no power.
					\item To deduce free will from the fathers is to interpret their contradictions in the worst possible way.
				\end{mynotes}
				\item The true church does not err.
				\begin{mynotes}
					\item Unholy men may have the appearance of religion, yet may be in the gravest theological error.
					\item Luther views his authorities as saints, not by the ``rule of faith'' but by the ``rule of charity.''
				\end{mynotes}
				\item All doctrines must be tested by Scripture.
				\begin{mynotes}
					\item Erasmus affirms the authority of Scripture but doubts its perspicacity---perhaps all Scripture is declared unclear, and theology devolves to the interpreting help of Rome.
					\item By the Holy Spirit, Scripture alone is powerful to judge both the faith of individual man and of the church.
				\end{mynotes}
				\item Scripture is maintained as clear.
				\begin{mynotes}
					\item Disputes are judged according to the Law (Deut.~17).
					\item Ps.~119: ``thy word,'' not ``thy Spirit alone,'' is ``a lamp to my feet \ldots''
					\item Perhaps the Scriptures are not everywhere uncertain, but only respecting e.g.~free will. But the Scriptures must be clear at every ``matter of concern.''
					\item The witness of history is that erudition is no excuse for confusion.
				\end{mynotes}
				\item Blindness of man vs.~clarity of Scripture
				\begin{mynotes}
					\item Men are by nature blind to Scripture's truth.
				\end{mynotes}
				\item Erasmus's use of history
				\begin{mynotes}
					\item Erasmus cites martyrs who seemed to prove free will, yet he would not allow them to even to think the Scripture was clear.
				\end{mynotes}
			\end{enumerate}
		\end{enumerate}
	\end{multicols}

\end{document}
