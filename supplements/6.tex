\documentclass[outline]{bookclub}

\begin{document}
	\setcounter{chapter}{5}
	\chapter{July 25, 2018: V.~(v)--VI.~(i)}
	\begin{multicols}{2}
		\begin{enumerate}
			\setcounter{enumi}{4}
			\item 
			\begin{enumerate}
				\setcounter{enumii}{4}
				\item God works hardening through the individual's evil will, not by creating evil anew.
				\item Of the hardening of Pharaoh
				\begin{mynotes}
					\item God hardened Pharaoh through Moses, whose righteousness excited Pharaoh's already rebellious will.
					\item God hardened Pharaoh by apparent longsuffering? (withholding rebuke)
					\begin{response}
						\item But God hardened Pharaoh during the \emph{plagues}, and displayed mercy during all the years of prior slavery.
						\item The hardening of Pharaoh was for the glory of God.
						\item God by his omnipotence move Pharaoh's will in the way it was naturally oriented, for it doubtful had power of its own.
					\end{response}
				\end{mynotes}
				\item Necessity of God's foreknowledge in Rom.~9:15
				\begin{mynotes}
					\item God's foreknowledge makes events necessary that actually come to pass.
					\item Is Rom.~9:20 a declaration of confusion?
					\begin{response}
						\item Luther: the rest of Rom.~9 makes clear that it is God who works in the will. Rather, a (prohibited) reckless investigation to harmonize God's will with human freedom amounts to attempting to deny God's sovereignty altogether.
					\end{response}
				\end{mynotes}
				\item Divine freedom implies human necessity.
				\begin{mynotes}
					\item It is not open to us to become something that God did not foreknow and is bringing about.
					\item That a merciful god would condemn so many is a stumbling block---for those who believe, a source of grace.
				\end{mynotes}
				\item Erasmus manipulates the smallest parts of Rom.~9 in order to confuse the whole.
				\item In the case of Judas there is necessity of consequence but not \emph{necessity of compulsion}.
				\item Mal.~1:2--3 (in Rom.~9)
				\begin{mynotes}
					\item Jacob and Esau were predestined before birth,  whence not free thereafter.
					\item God loves and hates according to his eternal and immutable nature.
					\item Faith and unbelief are not ours to render, but are because of the love and hatred of God.
				\end{mynotes}
				\item The potter and the clay
				\begin{mynotes}
					\item Erasmus: potter-clay figures refer to temporal affliction in the Prophets, but eternal reprobation in Paul.
					\begin{response}
						\item Luther: Paul is using the figure without reference to Jeremiah.
					\end{response}
					\item A man cannot purify himself, for all fault abides in the pots and no in the potter.
				\end{mynotes}
				\item Both God's justifying the ungodly and condemning the deserving defy human norms, yet Luther's objectors find only the latter disagreeable.
				\item Man's salvation is none of man's work and all of God's.
			\end{enumerate}
			\item \begin{enumerate}
				\item Gen 6:3 ``Spirit'' and ``flesh''
				\begin{mynotes}
					\item Erasmus responded that men were called corrupt (``flesh'') only at the time, and this passage does not establish the desperate sinfulness of all humans.
					\item Luther reads that Gen 6:3 depicts God's embittering man, though there was no need, as man is already fallen.\footnote{Honestly I wasn't sure what he meant here, probably because I'm not too familiar with Luther's earlier works or Erasmus's entire response. I personally find both readings pretty difficult.}
				\end{mynotes}
			\end{enumerate}
		\end{enumerate}
	\end{multicols}

\end{document}
