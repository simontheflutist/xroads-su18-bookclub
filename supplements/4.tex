\documentclass[outline]{bookclub}

\begin{document}
	\setcounter{chapter}{2}
	\chapter{July 10, 2018: IV.~(i)--(viii)}
	\begin{multicols}{2}
		\begin{enumerate}
			\setcounter{enumi}{3}
			\item 
			\begin{enumerate}
				\item Definition of ``free-will''
				\begin{mynotes}
					\item Erasmus defines free will as the human power to ``apply itself'' to choose (or decline) eternal salvation.
					\item Luther accuses Erasmus of semantic bait-and-switch: of arguing for a will that is merely able to \emph{choose} (``vertible/mutable will'') but concluding a will that is able to choose \emph{of its own accord} (``free will'').
					\item The will's purported freedom of self-direction (``apply itself''/``turn away'') comes between the will (as an immaterial entity) and its action (which is to \emph{will}).
					\item Paul/Isaiah: the things of God are dark and hidden to natural man.
					\item The self-moving, all-seeing will makes Erasmus even more wrong than the Pelagians\footnote{Adherents of an early heresy that man is born unaffected by ``original sin'' and can prefer good and God.} and Scholastics.\footnote{The medieval theological method in which Luther was trained. Scholastics believed that at least man's \emph{reason} is intact and can discover something of God without grace or special revelation.}
					\item The will can do good only with grace's help (Lombard) and is otherwise bound (Augustine).
					\item Error of the Scholastics: does a stone which falls, but rises with human help, have free will?
				\end{mynotes}
				\item Choice between life and death in Ecclesiasticus~15:14--17 does not pertain to free will.\footnote{Luther doesn't even accept Ecclesiasticus as canon, yet refutes Erasmus on his own terms anyway.}
				\item Three views on free will.
				\begin{mynotes}
					\item Erasmus accepts \#1, rejecting \#2 and \#3:\begin{enumerate}[label=\arabic*.] 
						\item Man cannot will good without special grace.
						\item ``Free will'' can do nothing but sin.
						\item ``Free will'' is a meaningless idea, for all comes to pass by divine necessity.
					\end{enumerate}
					\item Luther will claim the three views, though they seem to progress in severity, are equivalent.
					\item \#1 contradicts Erasmus's initial definition/thesis in which the unaided will can choose salvation (a good).
					\item Erasmus accidentally claims the will is unfree (Luther's position).
					\item Luther speculates that Erasmus discusses a meta-agency of ``\,`willing' in the abstract'' to the will that abides above \emph{choice} between good and evil.
					\begin{response}
						\item Rejection: ``He that is not with me is against me'' (Matt.~12:30) and there is no free intermediate.
					\end{response}
					\item \begin{description}
						\item[\(1\implies 2\)] Impotence toward good implies bondage to sin.
						\item[\(2\implies 3\)] From Erasmus's definition of ``free will,'' which no longer appears ``free.''
						\item[\(3\implies 1\)] If man can will nothing without help, he cannot will good without help. Help towards willing good is grace.\footnote{Luther doesn't argue this case explicitly, but I think it's sufficiently implied.}
					\end{description}
				\end{mynotes}
				\item Choice in Ecclus.~(contd.)
				\begin{mynotes}
					\item ``If thou art willing to keep the commandments'' conditions blessing on the will of man without speaking to its freedom.
				\end{mynotes}
				\item Command does not imply ability.
				\begin{mynotes}
					\item Subjunctives in the law are not pronouncements of man's ability, but edifying proofs of his bondage. Man recognizes his own bondage by grace to turn to God.
				\end{mynotes}
				\item Freedom would imply plenary ability.
				\begin{mynotes}
					\item Erasmus's hermeneutic on Ecclus.~does not exclude the Pelagian reading that man is not fallen at all.
				\end{mynotes}
				\item Can man overcome his sinful will by force?
				\begin{mynotes}
					\item ``Under thee shall be the desire of sin, and thou shalt rule over it'' (Gen.~4:7).
					\item This is not a promise, but a command, just as in the Decalogue.
				\end{mynotes}
				\item ``Choose what is good'' in Deut.~30:19
				\begin{mynotes}
					\item ``By the law is the \emph{knowledge} of sin,'' not the \emph{avoidance}.
					\item Erasmus imagines either a man who can do what he is commanded, or is frustrated he cannot. Yet the blindness of fallen man under Satan's power is in that though dead, bound, and wretched, he believes himself alive, free, and happy.
				\end{mynotes}
			\end{enumerate}
		\end{enumerate}
	\end{multicols}

\end{document}
