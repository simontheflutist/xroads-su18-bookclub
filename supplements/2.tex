\documentclass[outline]{bookclub}

\begin{document}
	\setcounter{chapter}{1}
	\chapter{June 22, 2018: I.--II. (vi)}
	\begin{multicols}{2}
		\begin{enumerate}
			\item 
%			\begin{mynotes}
%				\item 
				Luther concedes that Erasmus is both more cultured and more intelligent than he is, and is surprised by the poor quality of Erasmus's theology.
%				\item 
				He claims to have Scripture on his side, whereas Erasmus's work is ``slippery and evasive.''
%				\item 
				Nevertheless, he is grateful that Erasmus has given him time and reason to refine his own views.
%			\end{mynotes}
			\item 
			\begin{enumerate}
				\item \emph{Assertions} considered harmful---an ``undogmatic temper'' preferred?
				\begin{mynotes}
					\item Necessity of assertions: ``confess'' (Rom.~10, Matt.~10:32),
					``reason'' (1 Pet.~3:15), conviction of the Holy Spirit (John~16:8), ``rebuke' (2 Tim.~4:2)
					\item Supremacy of Scripture: ``let the others judge'' (1 Cor.~14:29)---though Erasmus does not truly submit to the Church either: ``whether I follow or not.''
				\end{mynotes}
				\item Some doctrines are unclear, but others are plain?
				\begin{mynotes}
					\item At face value Luther agrees, for some matters are truly mystery.
					\item Though \emph{passages} might be unclear, \emph{content} is clear and confusion is due to man's blindness alone.
				\end{mynotes}
				\item ``Free will'' is a non-essential doctrine?
				\begin{mynotes}
					\item Rather it addresses 1) why we sin and 2) how we obtain salvation.
					\item Claims of irrelevance are by nature \emph{assertions} that Erasmus fails to eschew.
					\item Catch-22: we cannot ``strive with all our might'' and thereby obtain ``the mercy of God, without which man's will and endeavour is ineffective.'' 
					\item Na\"ive confidence in human will leads to self-righteousness and a repudiation of God's nature and power:
					\((\text{knowledge of self})\iff (\text{knowledge of God})\).
					\item Overall, Erasmus is not so much outright wrong as badly conflicted and inconsistent.
				\end{mynotes}
				\item God does not foreknow contingently.
				\begin{mynotes}
					\item God's will must be identified with his foreknowledge, and it is as immutable as God himself.
					\item A thing might be contingent, but God's bringing it about may be necessary; \((x\ \text{is done contingently}) \nLeftrightarrow \left(x\ \text{contingent}\right)\).
					\item ``Necessity'' is an awkward term because it suggests compulsion, whereas our wills (and God's) evidently proceed as they please.
					\item The Sophists, half-affirmed? A distinction between ``necessity of consequence'' and ``necessity of the consequent'' reduces to distinction between God and creature.
					\item Predestination is naturally revealed even to the wise of this world---cf.~Virgil.
				\end{mynotes}
				\item Significance of the previous point.
				\begin{mynotes}
					\item Only if we believe God acts ``necessarily and immutably'' can we trust his promises.
					\item The questioning of the Sophists is rejected, in favor of interpreting Scripture anew.
				\end{mynotes}
				\item Certain truths are better unheard?
				\begin{mynotes}
					\item Sacred ``duty and doctrine'' impose themselves upon us and are not ours to shut out.
					\item Erasmus is giving dogmatic opinion about free will and fails to heed his own advice on illegal subjects.
					\item Objection: God is in all places, in a beetle's hole etc.~as well as Heaven, but we suppress these facts in order to facilitate worship.
					\begin{enumerate}[label=(\arabic*)]
						\item Indeed all are true, and all \emph{prima facie} truths can be both preached and heard sinfully.
						\item Response: God subjected himself to humanity, even to death, for our benefit and his glory.
					\end{enumerate}
					\item Objection: there are in fact three gods. (Response: no.)
					\item Erasmus's doctrine binds conscience as it was not meant to be bound.
					\item Objection: truth is by nature lawful but not always helpful.
					\begin{enumerate}[label=(\arabic*)]
						\item Response: Luther's particular truth is of so great consequence that it is surpassed by no earthly priority. Truth must not be sold for a false peace.
					\end{enumerate}
					\item Objection: ``some diseases may be born with less harm than they can be cured with.''
					\begin{enumerate}[label=(\arabic*)]
						\item Response: eternal damnation is worse than temporal dispute.
					\end{enumerate}
					\item The Gospel does not bring about human evil, but rather reveals it.
					\item Objection: wrong decisions of church councils ought not to be laid bare, in deference for authority.
					\begin{enumerate}[label=(\arabic*)]
						\item Scripture, not the Church, is the final judge of all things.
					\end{enumerate}
				\end{mynotes}
			\end{enumerate}
		\end{enumerate}
	\end{multicols}

\end{document}
