\documentclass[outline]{bookclub}

\begin{document}
	\setcounter{chapter}{4}
	\chapter{July 21, 2018: IV.~(ix)--V.~(iv)}
	\begin{multicols}{2}
		\begin{enumerate}
			\setcounter{enumi}{3}
			\item 
			\begin{enumerate}
				\setcounter{enumii}{8}
				\item Law and Gospel.
				\begin{mynotes}
					\item Seeming imperatives juxtapose law and grace---``turn ye'' is a proclamation of law, ``I will turn'' of grace.
					\item ``I desire not the death of a sinner''---showing God's mercy for the penitent. The unconverted have no interest in God's mercy. (Neither should this be taken to mean that God prohibits the death of sinners.)
				\end{mynotes}
				\item The hidden will of God ``works life, and death, and all in all,'' and should not be questioned.
				\item Obligation does not imply ability in Deut.~30:11--14.
				\begin{mynotes}
					\item ``For this commandment which I command thee this day, it is not hidden from thee, neither is it far off.
					It is not in heaven, that thou shouldest say, Who shall go up for us to heaven, and bring it unto us, that we may hear it, and do it?
					Neither is it beyond the sea, that thou shouldest say, Who shall go over the sea for us, and bring it unto us, that we may hear it, and do it?
					But the word is very nigh unto thee, in thy mouth, and in thy heart, that thou mayest do it.''
					\item Erasmus would have this mean not only that we are free, but we are naturally inclined to good.
					\item Luther reads that this is about the plainness of the Law, so that it might expose human inability and unrighteousness.
				\end{mynotes}
				\item Jerusalem would not turn back, though it could? Matt.~23:37.
				\begin{mynotes}
					\item It must not be help contradictory that God grieves the perdition of the ungodly even though he still permits them to be so.
					\item Paul and Isaiah warn against searching certain mysteries of God's will, since as a tactic for denying them.
				\end{mynotes}
				\item ``If thou wilt\ldots'' (etc.)
				\item Good and bad works in the New Testament
				\begin{mynotes}
					\item The New Testament consists of ``promises and exhortations'' whereas the Old Testament ``laws and threats.'' \footnote{Here Luther finally makes his reading of old vs.~new explicit that permeates so much of earlier discussion of law and prophecy. This reading provided for a sharper break with Rome, but I think it is almost completely mistaken.}
					\item Therefore the exhortation of the New Testament is addressed to people who already know their own inability and have been converted by the Holy Spirit.
				\end{mynotes}
				\item Reward grounded in God's promise, not man's merit
				\begin{mynotes}
					\item ``Necessity has neither merit or reward'' and thus New Testament directives lose force if our hypothetical obedience is mere necessity.
					\begin{response}
						\item ``---of compulsion'': this is certainly the case.\footnote{For example, a programmer shouldn't ``reward'' or ``punish'' his machine---all software events happens necessarily.}
						\item ``---of immutability'': in this event good and evil are done \emph{willingly} though the will is fixed.
					\end{response}
					\item ``Reward and punishment follow \emph{naturally} and \emph{necessarily}.''
					\begin{response}
						\item ``Naturally'' suggests \emph{consequence}, rather than \emph{worthiness} (``of merit'').
						\item Christians logically cannot attempt to \emph{merit} what is already theirs.
					\end{response}
				\end{mynotes}
				\item Necessity and moral responsibility
				\begin{mynotes}
					\item Our fruits as as much ``ours'' as our hands and feet.
					\item ``To them he gave power to \emph{become}'': not ``power'' of a free will, but a passive privilege of \emph{becoming} (sans \emph{doing}).
				\end{mynotes}
			\end{enumerate}
			\item \begin{enumerate}
				\item Erasmus evaluates Pharaoh (Ex.) and Esau (Mal.), allegedly denying free will.
				\item Erasmus discovers figures of speech in Scripture where they might be dubious.
				\item Hardening of Pharaoh
				\begin{mynotes}
					\item Erasmus reads ``\,`I will harden the heart of Pharaoh'\,'' as God's passive hardening by \emph{patience}: God's evident agency is removed.
					\item Indirect figures of God's passivity turn mercy and wrath backwards---patience becomes wrath and affliction becomes mercy.\footnote{Luther takes it upon himself to respond according to Erasmus's weird reading. This part is really convoluted, and you don't encounter Erasmus's view much anymore, so I left it out.}
					\item ``Though God does not make sin, yet he does not cease to form and multiply our nature, from which the Spirit has been withdrawn and which sin has impaired.''
				\end{mynotes}
				\item God's method of working evil in man
				\begin{mynotes}
					\item God moves wicked man as a horse with three feet, or a saw-toothed ax.
				\end{mynotes}
			\end{enumerate}
		\end{enumerate}
	\end{multicols}

\end{document}
